\documentclass[10pt,a4paper]{article}
\usepackage[margin=1in]{geometry}
\usepackage{todonotes}
\setuptodonotes{color=blue!25,tickmarkheight=1.5mm,size=\small}

\usepackage[T1]{fontenc}
\usepackage{amssymb}
\usepackage{amsthm}
\usepackage{physics}
\usepackage[dvipsnames]{xcolor} %colors

\usepackage[draft,inline,nomargin]{fixme} \fxsetup{theme=color} % Comments
\definecolor{jacolor}{RGB}{200,40,0}
\FXRegisterAuthor{ja}{aja}{\color{jacolor}JA}

\usepackage{hyperref}
\hypersetup{
colorlinks=true,
linkcolor=blue,
filecolor=blue,      
citecolor=blue,
urlcolor=blue,
pdftitle={},
pdfauthor=author={},
}

\usepackage[
backend=bibtex,
style=phys,
maxbibnames=5,
biblabel=brackets,
hyperref=true,
arxiv=abs,
eprint=true,
url=false,
doi=false
]{biblatex}
\AtEveryBibitem{\clearfield{note}}
\AtEveryBibitem{\clearfield{pubstate}}  % Remove pubstate from all entries
\addbibresource{symmetries_and_chaos.bib} 
\title{Survival probability}
\begin{document}
\maketitle
\tableofcontents

\pagebreak

\section{Question}
Why is the correlation hole not affected by the precense of a high degree
of symmetries in Ref.~\cite{delacruz_2020_quantum}?

\section{Mostly ideas}
\subsection{Survival probability in a system with the simplest symmetric structure}
The survival probability is defined as 
\begin{equation}
S_p(t)
= 
\abs{\braket{\psi_0}{\psi(t)}}^2,
\end{equation}
Consider that $H \ket{\phi_n} = E_n \ket{\phi_n}$, then the survival probability
$S_p$ yields
\begin{equation}
S_p(t) 
= 
\abs{
\sum_{k}
\abs{\braket{\psi_0}{\phi_k}}^2
e^{-i E_k t}
}^2.
\end{equation}
Let us consider a symmetry of $H$ under an arbitrary operator $\Pi$ such 
that $H = H_1 \oplus H_2$---for instance, you may consider $\Pi$ to be the 
reflection operator. Then, we can relabel the energies $E_n$ such that 
$\{E_n\}_{n=1}^p$ and $\{E_n\}_{n=p+1}^N$ are the spectra of $H_1$ and 
$H_2$, respectively. $N$ denotes the dimension of $H$, $p$ the dimension of 
$H_1$ and $N-p$ is the dimension of $H_2$. After this relabeling, $S_p(t)$
takes the form:
\begin{align}
S_p(t) 
&=
\abs{
\sum_{k = 1}^p
\abs{\braket{\psi_0}{\phi_k}}^2
e^{-i E_k t}
+
\sum_{l = p+1}^N
\abs{\braket{\psi_0}{\phi_l}}^2
e^{-i E_l t}
}^2 \\
%
&= 
\abs{
\sum_{k = 1}^p
\abs{\braket{\psi_0}{\phi_k}}^2
e^{-i E_k t}
}^2 
+
\abs{
\sum_{l = p+1}^N
\abs{\braket{\psi_0}{\phi_l}}^2
e^{-i E_l t}
}^2 \\
&\quad +
2\Re{
\sum_{k=1}^{p+1}\sum_{l = p+1}^N
\abs{\braket{\psi_0}{\phi_k}}^2
\abs{\braket{\psi_0}{\phi_l}}^2
e^{-i(E_k - E_l) t}
}.
\end{align}
The first two terms are recognized as the survival probability in the two
symmetric subspaces. The third term God knows what the hell it is, but it 
is the term explaining why the correlation whole survives (or not).

\subsection{Survival probability as function of level spacings}
Let us rewrite the survival probability as follows:
\begin{align}
S_p(t) 
&= 
\abs{
\sum_{k}
\abs{\braket{\psi_0}{\phi_k}}^2
e^{-i E_k t}
}^2 
\\
%
&= 
\sum_{k,l}
\abs{\braket{\psi_0}{\phi_k}}^2
\abs{\braket{\psi_0}{\phi_l}}^2
e^{-i (E_k - E_l) t} \\
%
&= 
\sum_{k=1}^N
\abs{\braket{\psi_0}{\phi_k}}^2
\abs{\braket{\psi_0}{\phi_k}}^2
+ 2\Re\Bigg[
\sum_{k=1}^{N-1}
\abs{\braket{\psi_0}{\phi_k}}^2
\abs{\braket{\psi_0}{\phi_{k+1}}}^2
e^{-i (E_k - E_{k+1}) t} 
\nonumber \\
%
&\quad\, + 
\sum_{k=1}^{N-2}
\abs{\braket{\psi_0}{\phi_k}}^2
\abs{\braket{\psi_0}{\phi_{k+2}}}^2
e^{-i (E_k - E_{k+2}) t} 
+ \ldots
\nonumber \\
%
&\quad\, + 
\abs{\braket{\psi_0}{\phi_N}}^4
e^{-i (E_1 - E_N) t} \Bigg]. 
\label{eq:Sp:spacings}
\end{align}
In this form, it is explicit that the survival probability can be written
as a sum of the contributions given by higher order spacings. From 
Ref.~\cite{tekur_2020_symmetry} we learned not only that RMT's predictions
can be verified even a non-desymmetrized system, but also that higher 
order spacing ratios exhibit correlations that match RMT's predictions.
Very recently, a work on the distribution of higher order spacings
for the circular gaussian ensembles appeared on the 
arXiv~\cite{rout_2025_higherorder} 
\janote{I think these results we can use them to study
the behavior of each term in Eq.~\eqref{eq:Sp:spacings}. My guess is that 
the most relevant contributions to $S_p(t)$ come from the higher order 
spacing correlations}.

Let us consider two identical spectra $\{E^{(1)}_1, E^{(1)}_2, E^{(1)}_3\}$
and $\{E^{(2)}_1, E^{(2)}_2, E^{(2)}_3\}$ (that is, $E^{(1)}_i = E^{(2)}_i$).
The whole ordered spectrum will be 
$\{E^{(1)}_1, E^{(2)}_1, E^{(1)}_2, E^{(2)}_2, E^{(1)}_3, E^{(2)}_3\}$.
Then, compute the spacings:
\begin{align}
s^{(1)}_1 &= E^{(2)}_1 - E^{(1)}_1 = 0\\
s^{(2)}_1 &= E^{(1)}_2 - E^{(1)}_1 \\
s^{(3)}_1 &= E^{(2)}_2 - E^{(1)}_1 \\
s^{(4)}_1 &= E^{(1)}_3 - E^{(1)}_1 \\
s^{(5)}_1 &= E^{(2)}_3 - E^{(1)}_1 
\end{align}
The spacings of first order $s_n^{(1)}$ follow a Poissonian distribution, 
possibly a zero-inflated Poisson distribution as half the spacings will 
become zero. The second-order spacings become the first-order spacings 
of both subspaces \janote{I think the PDF here will be the convolution of 
two Wigner surmises}. The third-order spacings become a mix between first-
and second-order spacings \janote{maybe the convolution of three Wigner surmises?}. The fourth-order spacings will become second-order spacings from
both subspaces\janote{once again, the convolution of four Wigner surmises?}.
The fifth-order spacings become a mix between second- and third-order 
spacings\janote{you know the drill by now, the convolution of five Wigner 
surmises?}

\janote{I guess the convolution of many Wigner surmises should converge to
something...? Hence, after some order the PDF should be almost the same.
Something to ask Deepseek or chatGPT}

\janote{If all of this is, at the very least, a good approximation, we still 
have to take into account the coefficients $\abs{\braket{\psi_0}{\phi_i}}^2
\abs{\braket{\psi_0}{\phi_{i+k}}}^2$.
Therefore, we may consider the case where all coefficients are equal, case
in which I'm almost sure the $S_p(t)$ becomes the spectral form factor. If it 
is indeed the case we should expect a dip-ramp-plateau behavior, the ramp being 
the hallmark of long-range correlations of a spectrum's chaotic system.
}

\appendix

\section{Average of $\sum_k e^{-i s_k t}$ for Wigner-Surmise Spacings}

We consider the nearest-neighbor spacings $s_k$ distributed according to the \textbf{Wigner surmise}:
\[
p(s) = \frac{\pi s}{2} e^{-\pi s^2 / 4}, \quad s \geq 0.
\]

We wish to compute:
\[
\left\langle \sum_k e^{-i s_k t} \right\rangle.
\]

\subsection*{Assumption}
Assuming the spacings are independent and identically distributed (i.i.d.) with the Wigner surmise distribution:
\[
\left\langle \sum_k e^{-i s_k t} \right\rangle \approx (N-1) \left\langle e^{-i s t} \right\rangle.
\]

\subsection*{Characteristic Function}
The characteristic function is:
\[
\phi(t) = \int_0^\infty e^{-i s t} \, p(s) \, \dd s = \int_0^\infty e^{-i s t} \cdot \frac{\pi s}{2} e^{-\pi s^2 / 4} \dd s.
\]

This integral evaluates to:
\[
\phi(t) = 1 - t e^{-t^2/\pi} \left( \mathrm{erfi}\left( \frac{t}{\sqrt{\pi}} \right) + i \right),
\]
where $\mathrm{erfi}(z) = \frac{2}{\sqrt{\pi}} \int_0^z e^{u^2} \dd u$ is the imaginary error function.

\subsection*{Final Result}
\[
\boxed{\left\langle \sum_k e^{-i s_k t} \right\rangle \approx (N-1)\left[1 - t e^{-t^{2}/\pi}\left(\mathrm{erfi}\left(\frac{t}{\sqrt{\pi}}\right) + i\right)\right]}
\]

\subsection*{Remarks}
\begin{itemize}
    \item This result assumes i.i.d. spacings (a common approximation using the Wigner surmise)
    \item In full RMT (GOE), spacings are not independent, but this captures key features
    \item For $t = 0$: $\phi(0) = 1$ as expected
    \item For small $t$: $\phi(t) \approx 1 - i t - \frac{2}{\pi} t^2 + \cdots$
\end{itemize}

\section{df}
For a quantum system with Poissonian level statistics, the nearest-neighbor spacings $s_k$ are independent and identically distributed with the exponential distribution:
\[
p(s) = e^{-s}, \quad s \geq 0,
\]
where the mean spacing is normalized to unity.

We wish to compute the average:
\[
\left\langle \sum_k e^{-i s_k t} \right\rangle.
\]

Assuming the spacings are independent, this becomes:
\[
\left\langle \sum_k e^{-i s_k t} \right\rangle = (N-1) \left\langle e^{-i s t} \right\rangle,
\]
where the characteristic function is:
\[
\phi(t) = \left\langle e^{-i s t} \right\rangle = \int_0^\infty e^{-i s t} e^{-s} \dd{s} = \int_0^\infty e^{-s(1 + i t)} \dd{s}.
\]

Evaluating this integral gives:
\[
\phi(t) = \frac{1}{1 + i t}.
\]

Therefore, the final result is:
\[
\boxed{\frac{N-1}{1+it}}
\]

\section{A possibly useful integral}
\begin{equation}
\expval{e^{-ist}} =
\frac{1}{2}
e^{-\frac{1}{2} t \left(\sigma ^2 t+2 i\right)} 
\qty[
1 + \erf\qty(\frac{1 - i\sigma^2 t}{\sqrt{2}\sigma})
]
\end{equation}

\printbibliography
\end{document}