\documentclass[10pt,a4paper]{article}
\usepackage[T1]{fontenc}
\usepackage{amssymb}
\usepackage{amsthm}
\usepackage{physics}
\title{Survival probability}
\begin{document}
\maketitle

The survival probability is defined as 
\begin{equation}
S_p(t)
= 
\abs{\braket{\psi_0}{\psi(t)}}^2,
\end{equation}
Consider that $H \ket{\phi_n} = E_n \ket{\phi_n}$, then the survival probability
$S_p$ yields
\begin{equation}
S_p(t) 
= 
\abs{
\sum_{k}
\abs{\braket{\psi_0}{\phi_k}}^2
e^{-i E_k t}
}^2.
\end{equation}
Let us consider a symmetry of $H$ under an arbitrary operator $\Pi$ such 
that $H = H_1 \oplus H_2$---for instance, you may consider $\Pi$ to be the 
reflection operator. Then, we can relabel the energies $E_n$ such that 
$\{E_n\}_{n=1}^p$ and $\{E_n\}_{n=p+1}^N$ are the spectra of $H_1$ and 
$H_2$, respectively. $N$ denotes the dimension of $H$, $p$ the dimension of 
$H_1$ and $N-p$ is the dimension of $H_2$. After this relabeling, $S_p(t)$
takes the form:
\begin{align}
S_p(t) 
&=
\abs{
\sum_{k = 1}^p
\abs{\braket{\psi_0}{\phi_k}}^2
e^{-i E_k t}
+
\sum_{l = p+1}^N
\abs{\braket{\psi_0}{\phi_l}}^2
e^{-i E_l t}
}^2 \\
%
&= 
\abs{
\sum_{k = 1}^p
\abs{\braket{\psi_0}{\phi_k}}^2
e^{-i E_k t}
}^2 
+
\abs{
\sum_{l = p+1}^N
\abs{\braket{\psi_0}{\phi_l}}^2
e^{-i E_l t}
}^2 \\
&\quad +
2\Re{
\sum_{k=1}^{p+1}\sum_{l = p+1}^N
\abs{\braket{\psi_0}{\phi_k}}^2
\abs{\braket{\psi_0}{\phi_l}}^2
e^{-i(E_k - E_l) t}
}.
\end{align}
The first two terms are recognized as the survival probability in the two
symmetric subspaces. The third term God knows what the hell it is, but it 
is the term explaining why the correlation whole survives (or not).

Let us rewrite the survival probability as follows:
\begin{align}
S_p(t) 
&= 
\abs{
\sum_{k}
\abs{\braket{\psi_0}{\phi_k}}^2
e^{-i E_k t}
}^2 
\\
%
&= 
\sum_{k,l}
\abs{\braket{\psi_0}{\phi_k}}^2
\abs{\braket{\psi_0}{\phi_l}}^2
e^{-i (E_k - E_l) t} \\
%
&= 
\sum_{k=1}^N
\abs{\braket{\psi_0}{\phi_k}}^2
\abs{\braket{\psi_0}{\phi_k}}^2
+ 2\Re\Bigg[
\sum_{k=1}^{N-1}
\abs{\braket{\psi_0}{\phi_k}}^2
\abs{\braket{\psi_0}{\phi_{k+1}}}^2
e^{-i (E_k - E_{k+1}) t} 
\nonumber \\
%
&\quad\, + 
\sum_{k=1}^{N-2}
\abs{\braket{\psi_0}{\phi_k}}^2
\abs{\braket{\psi_0}{\phi_{k+2}}}^2
e^{-i (E_k - E_{k+2}) t} 
+ \ldots
\nonumber \\
%
&\quad\, + 
\abs{\braket{\psi_0}{\phi_N}}^4
e^{-i (E_1 - E_N) t} \Bigg].
\end{align}
In this form, it is explicit that the survival probability can be written
as a sum of the contributions given by higher order spacings. 
\end{document}