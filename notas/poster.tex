%!TEX root = main.tex

\section{Poster}
\subsection{Main goal}
The main goal is to develop an intuitive understanding of the 
correlations associated with higher-order spacings in the Bose-Hubbard (BH) 
model, specifically within the interval for $J/U$ within which the BH model 
is transitioning from Poisson to Wigner-Dyson 
statistics as suggested by the short-range 
correlation indicators we have previously examined, namely the mean level 
spacing ratio $\ev{r}$ and the Kullback-Leibler (KL) divergence. 
More precisely, we aim to test the hypothesis that long-range correlations emerge gradually, akin to the gradual breakdown of tori in classical chaotic systems as the chaotic parameter is increased.

\subsection{To-do}
\begin{itemize}
\item Compute the mean spacing ratios of order $k$ for the BH model, 
a sufficiently large matrix sampled from the GOE, and random i.i.d 
uniformly sampled numbers (Poisson). For this, we want to test if the 
mean level spacing ratio of order $k=1$ is that of Poisson statistics, but
at higher orders it is near the values of GOE.
\janote{It is not clear for me if there is a difference using the two 
definitions of the ratios (the one that is between 0 and 1, and the 
other one), so it is safer to explore both of them}

\item To compute the KL divergence, we require the actual probability 
distribution of higher-order spacings for the GOE. I do not recall 
encountering this distribution in the literature; however, it may be 
discussed in Ref.~\cite{rout_2025_higherorder}. So one task is to verify 
this paper. Additionally, consider consulting chatGPT. If there 
is indeed no known probability density function (PDF), we will need to 
fit a PDF to the histogram of the data obtained by sampling a GOE matrix 
and subsequently compute the KL 
divergence with respect to the fitted PDF.
\end{itemize} 

%Compare numerically the statistics of higher-order spacings (and ratios if time 
%allows it) of the BH model and matrices sampled from the GOE. 
%
%The spacings of order $k$ $s^{(k)}$ 
%of integrable systems follow a Gamma distribution 
%\begin{equation}
%P_{\mathrm{Poisson}}(s^{(k)})
%= 
%\frac{s^{k-1} e^{-s}}{(k-1)!}
%\end{equation}