%!TEX root = main.tex

\section{Poster}
\subsection{Main goal}
The primary objective is to develop an intuitive understanding of the 
correlations associated with higher-order spacings in the Bose-Hubbard (BH) 
model, specifically within the parameter regime characterized by the 
transition from Poisson to Wigner-Dyson statistics suggested by the short-range 
correlation indicators we have previously examined, namely the mean level 
spacing ratio $\ev{r}$ and the Kullback-Leibler (KL) divergence. 
More precisely, I aim to test the hypothesis that long-range correlations emerge gradually, akin to the gradual breakdown of tori in classical chaotic systems as the chaotic parameter is increased.

\subsection{What to do}
Compare the mean spacing ratios of order $k$ of a GOE matrix (sufficiently big)
with that obtained from the BH model. That is, plot $k$ in the horizontal
axis, and $\ev{r^{(k)}}$ in the vertical axis. I would expect that if the 
system is chaotic the curve of GOE and of that of the BH model are the same.
But in the transition between integrable and chaotic, maybe we will observe 
deviations from the GOE? 

%Compare numerically the statistics of higher-order spacings (and ratios if time 
%allows it) of the BH model and matrices sampled from the GOE. 
%
%The spacings of order $k$ $s^{(k)}$ 
%of integrable systems follow a Gamma distribution 
%\begin{equation}
%P_{\mathrm{Poisson}}(s^{(k)})
%= 
%\frac{s^{k-1} e^{-s}}{(k-1)!}
%\end{equation}