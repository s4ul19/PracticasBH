\documentclass[10pt,a4paper]{article}
\usepackage[margin=1in]{geometry}
\usepackage{todonotes}
%\setuptodonotes{inline,color=blue!25,tickmarkheight=25mm,size=\small}
\setuptodonotes{
  inline,
  color=blue!25,           % Background color
  backgroundcolor=blue!25, % Same as color (alternative name)
  size=\small,            % \tiny, \scriptsize, \footnotesize, \small, \normalsize, etc.
  tickmarkheight=10mm,     % Height of margin marker
}

\usepackage[T1]{fontenc}
\usepackage{amssymb}
\usepackage{amsthm}
\usepackage{physics}
\usepackage[dvipsnames]{xcolor} %colors
\usepackage{float} % <- importante
\usepackage[draft,inline,nomargin]{fixme} \fxsetup{theme=color} % Comments
\definecolor{jacolor}{RGB}{200,40,0}
\FXRegisterAuthor{ja}{aja}{\color{jacolor}JA}

\usepackage{hyperref}
\hypersetup{
colorlinks=true,
linkcolor=blue,
filecolor=blue,      
citecolor=blue,
urlcolor=blue,
pdftitle={},
pdfauthor=author={},
}

\usepackage[
backend=bibtex,
style=phys,
maxbibnames=5,
biblabel=brackets,
hyperref=true,
arxiv=abs,
eprint=true,
url=false,
doi=false
]{biblatex}
\AtEveryBibitem{\clearfield{note}}
\AtEveryBibitem{\clearfield{pubstate}}  % Remove pubstate from all entries
\addbibresource{ref.bib} 
\graphicspath{{img/}}
\newcommand{\eref}[1]{eq.~(\ref{#1})} 
\newcommand{\sref}[1]{sec.~\ref{#1}}
%\newcommand{\tref}[1]{table~\ref{#1}}
\newcommand{\Eref}[1]{Eq.~(\ref{#1})} 
\newcommand{\Sref}[1]{Sec.~\ref{#1}}
\newcommand{\Appref}[1]{Appendix~\ref{#1}}
\newcommand{\Fref}[1]{Fig.~\ref{#1}}  
\newcommand{\Tref}[1]{Table~\ref{#1}}
\graphicspath{{img/}}
\begin{document}
\begin{titlepage}
    \centering
    \vspace*{1cm}
    
    {\huge\bfseries Informe de Práctica final: \par}
    \vspace{0.5cm}
    {\Large Caos cuántico en el modelo de Bose Hubbard. \par}
    
    \vspace{1.5cm}
    
    {\Large\textit{Saúl Nájera Allara} \par}
    
    \vspace{3cm}
    
    % Puedes incluir un logo aquí si quieres:
    % \includegraphics[width=0.4\textwidth]{logo.png} 
    
    \vfill
    
    {\large Universidad de San Carlos de Guatemala, USAC. \par}
    {\large Escuela de ciencias físicas y matemáticas, ECFM \par}
    \vspace{0.5cm}
\end{titlepage}
\section{Introducción:}
\section{Marco Teórico:}
\begin{enumerate}
    \item Bose Hubbard.
    El modelo se puede emplear tanto para fermiones como para bosones. En este trabajo se modelará para bosones. El Hamiltoniano de este modelo es:
\begin{equation}
\hat{H}
= -\frac{J}{2} \sum_{\langle i,j\rangle} \!\left(\hat{a}_i^{\dagger}\hat{a}_j + \hat{a}_j^{\dagger}\hat{a}_i\right)
+ \frac{U}{2}\sum_{i}\hat{n}_i(\hat{n}_i - 1).
\end{equation}
Los términos $\hat{a}_i^{\dagger},\hat{a}_j$ son los operadores creación y aniquilación respectivamente y $\hat{n}_i$ es el operador número (cuenta el número de particulas en el sitio).
El parámetro $J$ describe la parte cinética del Hamiltoniano y dicta la probabilidad de que los bosones ``salten'' de un sitio a otro adyacente. El segundo parámetro $U$ describe la parte de interacción del Hamiltoniano, nos dice cómo  
es la fuerza de interacción entre bosones en mismo sitio~\cite{Zhang2010}. 
    \item Mean spacing Ratio (\textit{hablar brevemente de el spacing ratio y el problema de unfolding.})
Denotando por $E_n$ las energía ordenadas del espectro del Hamiltoniano, y el espaciamiento consecutivo de niveles de energía como 
$s_n=E_{n+1}-E_{n}$, se define la cantida~\cite{Atas_2013}:
\begin{equation}
	r_n=\frac{\min(s_n,s_{n-1})}{\max(s_n,s_{n-1})}.
\end{equation}
Esta cantidad permite observar la universalidad de RMT en los espectros de energía.
Estudiando la distribución de probabilidad asociada a las $r$'s, es posible determinar si la configuración está en un regímen integrable 
o caótico. Es importante tener presente la forma análitica de la distribución de probabilidad de los cocientes entre 
los espaciamientos de niveles del ensabamble GOE y de Poisson.
La distribución asociada a GOE es de la forma~\cite{Atas_2013}:
\begin{equation}
 P_{GOE}(r)=2\Theta(1-r)\cdot \qty(\frac{27}{8}\frac{r+r^{2}}{(1+r+r^{2})^{5/2}}),
\end{equation}
donde $\Theta(r)$ es la función de Heaviside.
Y la distribución de Poisson es de la forma~\cite{Tekur2020}:
\begin{equation}
    P_{Poisson}(r)=2\Theta(1-r)\cdot\qty(\frac{1}{(1+r)^{2}})
\end{equation}
Con estas expresiónes es posible extraer un $\ev{r}$ característico de cada distribucion.
En particular, para el ensamble GOE, dicho valor es de $\ev{r}_{GOE}=0.5307$~\cite{Atas_2013} 
y para la distribución de Poisson  un valor de $\ev{r}_{Poisson}=0.38629$~\cite{Atas_2013}. 
   \item Divergencia de Kullback-Leibler ($KL$): Es una medida de diferencia entre dos distribuciones de probabilidad. En este caso, se comparará la distribucion de probabilidad de los $r_n$ de la ecuación (2)
    respecto a la distribucion de probabilidad de la misma cantidad según el ensamble GOE. Viene dada por~\cite{Pausch2022}:
    \begin{equation}
KL(P,P_{GOE}) = \int_{0}^{1} P(x) \, \ln \left( \frac{P(x)}{P_{GOE}(x)} \right) \, dx.
    \end{equation}
    En esencia, es una medida de que tanto se aleja la distribución del sistema respecto a la distribución de GOE.   
\end{enumerate}
\section{Resultados:}
\textbf{Las resultados deben de ser recalculados \\ Mejorar las gráficas y probar n=m=10}
\begin{figure}[H]
    \centering
    \includegraphics[width=0.8\textwidth]{KLAndMean.png}
    \caption{Kl divergence y mean spacing de varias configuraciones J/U}
    \label{fig:KLAndMean}
\end{figure}
\begin{figure}[H]
    \centering
    \includegraphics[width=0.8\textwidth]{KLSolo.png}
    \caption{Kl divergence de varias configuraciones J/U}
    \label{fig:KL}
\end{figure}
\begin{figure}[H]
    \centering
    \includegraphics[width=0.8\textwidth]{MapSolo.png}
    \caption{Mean spacing de varias configuraciones J/U}
    \label{fig:Mean}
\end{figure}
\begin{figure}[H]
    \centering
    \includegraphics[width=0.8\textwidth]{BHeigenDist.png}
    \caption{Distribucion de eigenvalores de una configuracion (¿Cual?)}
    \label{fig:Eigen}
\end{figure}
\begin{figure}[H]
    \centering
    \includegraphics[width=0.8\textwidth]{GOEexamplePs.png}
    \caption{Spacing ratio de una configuracion (¿Cual?)}
    \label{fig:Spacing}
\end{figure}
\section{Discusión de resultados:}
\begin{enumerate}
\item Discutir sobre la figura que compara KL y Mean spacing Ratio, es importante ese resultado, pues ambas cantidades en esencia estan calculando algo distinto. Hay cierta universalidad en el diagnóstico.

\end{enumerate}
\printbibliography

\end{document}