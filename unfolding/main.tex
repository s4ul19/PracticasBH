\documentclass[10pt]{article}
\usepackage{amsmath, amssymb, physics}
\usepackage[margin=1in]{geometry}
\usepackage{booktabs}

\title{Why Unfolding is not necessary for ratio of consecutive level spacings}
\author{}
\date{}

\begin{document}
\maketitle

%\section{The Fundamental Problem: Non-Uniform Level Density}
%
%When we diagonalize a quantum Hamiltonian, we obtain eigenvalues $\{E_i\}$ 
%with a non-constant density of states $\rho(E)$. The naive level spacing 
%is defined as:
%%
%\begin{equation}
%s_i = E_{i+1} - E_i
%\end{equation}
%%
%The distribution of these raw spacings $P(s)$ reflects both:
%\begin{itemize}
%\item \textbf{Universal fluctuations}: Level repulsion, spectral rigidity
%\item \textbf{System-specific density}: $\rho(E)$ varies with energy
%\end{itemize}
%
%%Rewrite the following section to explain in more detail the unfolding procedure from the density of states
%\section{Unfolding: Removing System-Specific Density}
%
%We separate the cumulative level count into smooth and fluctuating parts:
%%
%\begin{equation}
%N(E) = \overline{N}(E) + N_{\text{fluc}}(E)
%\end{equation}
%%
%where $\overline{N}(E)$ is the smooth cumulative density. Unfolded levels 
%are defined as:
%%
%\begin{equation}
%\epsilon_i = \overline{N}(E_i)
%\end{equation}
%%
%The unfolded spacings become:
%%
%\begin{equation}
%\tilde{s}_i = \epsilon_{i+1} - \epsilon_i
%\end{equation}
%%
%Crucially, the \textbf{average} $\langle \tilde{s}_i \rangle = 1$ 
%throughout the spectrum, allowing comparison with Wigner's surmise 
%$P_{\text{Wigner}}(s)$ which assumes unit mean spacing.
\section{Unfolding: Removing System-Specific Density}

In quantum systems, the density of states often exhibits both a smooth 
background and superimposed fluctuations. To analyze the underlying 
statistical properties of the energy levels, it is essential to separate 
these components. This process is known as unfolding. Let us discuss how 
to perform the unfolding of a spectrum.

We begin defining the cumulative level count function $N(E)$, also known as
staircase function:
\begin{equation}
N(E) = \sum_i \Theta(E - E_i).
\end{equation}
One then needs to find a smooth function $\overline{N}(E)$ that approximates 
$N(E)$, by using techniques such as polynomial fitting or spectral 
averaging. The goal is to capture the average trend of energy levels 
distribution, filtering out the individual deviations.
Then, the unfolded levels, denoted as $\epsilon_i$, are 
defined as:
\begin{equation}
\epsilon_i = \overline{N}(E_i)
\end{equation}

This transformation ensures that the unfolded levels have a uniform 
average density. Consequently, the spacings between consecutive 
unfolded levels, $\tilde{s}_i$, are given by:

\begin{equation}
\tilde{s}_i = \epsilon_{i+1} - \epsilon_i
\end{equation}

An important property of the unfolded spacings is that their average, 
$\langle \tilde{s}_i \rangle$, is equal to 1 throughout the entire 
spectrum. This normalization is absolutely necessary for the comparison of the 
spacing distribution with the predictions of RMT.

\section{The Ratio Statistics: Automatic Scale Invariance}

Consider three consecutive levels: $E_{i-1}, E_i, E_{i+1}$. The ratio 
of consecutive spacings is defined as:
%
\begin{equation}
r_i = \frac{\min(s_i, s_{i-1})}{\max(s_i, s_{i-1})}
\end{equation}
%
where $s_i = E_{i+1} - E_i$ are the level spacings. This definition, which 
is not the only one, ensures $0 \leq r_i \leq 1$.
Now let's examine what happens under the unfolding transformation.

The unfolded spacings are:
%
\begin{align}
\tilde{s}_i 
&= 
\overline{N}(E_{i+1}) - \overline{N}(E_i) 
%\\
%
%&= 
%\rho(E_{i+1}) \cdot s_i - \rho(E) \cdot s_{i-1}
\end{align}
%
%\begin{equation}
%\tilde{s}_{i-1} = \epsilon_i - \epsilon_{i-1} = \overline{N}(E_i) 
%- \overline{N}(E_{i-1})
%\end{equation}
%
For sufficiently localized triplets, the density of states should be 
approximately constant we can approximate using the mean 
value theorem:
%
\begin{equation}
\tilde{s}_i \approx \rho(E_i) \cdot s_i, \quad 
\tilde{s}_{i-1} \approx \rho(E_i) \cdot s_{i-1}
\end{equation}
%
where $\rho(E) = \dv{\overline{N}}{E}$ is the local density of states.
Now compute the ratio of unfolded spacings:
%
\begin{equation}
\tilde{r}_i = \frac{\min(\tilde{s}_i, \tilde{s}_{i-1})}
{\max(\tilde{s}_i, \tilde{s}_{i-1})}
= \frac{\min(\rho(E_i)s_i, \rho(E_i)s_{i-1})}
{\max(\rho(E_i)s_i, \rho(E_i)s_{i-1})}
= \frac{\min(s_i, s_{i-1})}{\max(s_i, s_{i-1})} 
= r_i
\end{equation}
%
Therefore, we have proved that unfolding is not needed for computing the 
ratio of consecutive level spacings, since ratios are ``invariant'' under 
unfolding.

\end{document}