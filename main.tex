\documentclass[spanish,titlepage,table]{practicas}
%\usepackage[utf8]{inputenc}
\usepackage{parskip}
\usepackage[spanish]{babel}
\usepackage{hyperref}

%\directlua{tab = require("src/andpocalipse")}

\usepackage[draft,inline,nomargin]{fixme} \fxsetup{theme=color}
\definecolor{jacolor}{RGB}{200,40,0}\FXRegisterAuthor{ja}{aja}{\color{jacolor}JA}
\definecolor{accolor}{HTML}{008080}\FXRegisterAuthor{ac}{anac}{\color{accolor}AC}
\definecolor{cpcolor}{RGB}{0,200,40}\FXRegisterAuthor{cp}{ancp}{\color{cpcolor}CP}

%\usepackage[mathlines]{lineno}  \linenumbers \setlength\linenumbersep{3pt}


\decimalpoint
\hypersetup{
    colorlinks,
    allcolors=blue!70!black
}
\def\ASPOR{\phantom{Asesorado por:}\hspace{.6em}}

\institute{
    Universidad de San Carlos de Guatemala\\[.5em]
    Escuela de Ciencias Físicas y Matemáticas
}
\title{
    {\LARGE Prácticas:}\\[0.5em]
   Caos cuántico en el modelo de Bose-Hubbard \janote{Mucho texto, qué tal ``Caos cuántico en el modelo
de Bose-Hubbard''?}
}
\course{Licenciatura en Física Aplicada}
\author{Saúl Estuardo Nájera Allara}
\id{202107506}
\date{Septiembre de 2025}
\professor{
    M.Sc. José Alfredo de León (IF--UNAM),\\
    \ASPOR Ing. Rodolfo Samayoa (ECFM--USAC)
}

\begin{document}

\maketitle

\section{Descripción general de la institución}\label{sec:institution}
\subsection{Instituto de investigación de Ciencias Físicas y Matemáticas, USAC}
El Instituto de investigación de Ciencias Físicas y Matemáticas (ICFM) es la unidad de la Escuela de Ciencias Físicas y Matemáticas (ECFM) que promueve y realiza estudios avanzados en áreas científicas, fundamentales y aplicadas, de las ciencias físicas y matemáticas. El ICFM se proyecta como una plataforma regional de excelencia dedicada a la investigación y difusión del conocimiento en física y matemática. Las principales líneas de trabajo del ICFM son:
\begin{itemize}
    \item La investigación académica en ciencia básica y aplicada.
    \item La promoción de la investigación en ciencia básica y aplicada en el ámbito universitario.
    \item La difusión y divulgación del conocimiento generado por la investigación en ciencias físicas y matemáticas.
    \item La actualización continua de programas académicos de ciencias físicas y matemáticas.
\end{itemize}

%Seccion  actual de descripcion de trabajo
\section{Descripción general del proyecto}\label{sec:description}
\textbf{NEW ver.}
\\
El estudio estadístico del espectro de sistemas cuánticos es una herramienta de valiosa utilidad para el ámbito teórico, experimental y aplicado.
Se han distinguido dos tipos de propiedades: globales y locales.
Respecto a las propiedades locales, se han estudiado fluctuaciones del espectro de energía, es decir, se han observado desviaciones de las distribuciones de niveles de energía respecto de un comportamiento promedio. \cite{Bohigas_1984}
\newline
En un contexto clásico, el caos se define por la sensibilidad exponencial a las condiciones inicales 
(las trayectorias en el espacio de fase divergen a medida que transcurre el tiempo). En un contexto cuántico, en vista del principio de incertidumbre, no es posible tener 
trayectorias definidas y es mediante estadísiticas espectrales que se analiza la dinámica caótica del sistema.
Las fluctuaciones espectrales de sistemas cuánticos complejos es analizada mediante 
el marco teórico de la teoría de matrices aleatorias (RMT) en varias áreas de la física, 
tales como la física de materia condensada, física nuclear y física atómica.
Las fluctuaciones contienen señales distintivas para la identificación de diferentes fases observadas 
en sistemas físicos, por ejemplo para identificar límites entre integrabilidad o caoticidad en un sistema con análogo clásico,
determinación de fases de metalicidad o insulación, fases localizadas o termalizadas de sistemas de muchos cuerpos, estudio de regímenes en espectros nucleares, entre otros. \cite{Tekur2020} \newline
El presente consenso es que las fluctuaciones espectrales de sistemas cuánticos complejos presentan 
una repulsión entre sus niveles de energía que están en concordancia con alguno de los ensambles de matrices 
aleatorias, en los que se resaltan los ensambles GOE (Gaussian ortogonal ensamble), GUE (Gaussian unitary ensamble) y GSU (Gaussian simplectic ensamble).
\newline
Para hamiltonianos cuánticos cuya contraparte clásica es integrable, la conjetura Berry-Tarbor establece que 
la estadística de los niveles de energía sigue una estadística de Poisson, mientras que para hamiltonianos 
cuya contraparte clásica sea caótica, la conjetura BGS (Bogigas, Gianoni, Schmit) establece que la estadística de los niveles de energía es descrita 
por alguno de los ensambles de RMT.\cite{Atas_2013}
Es de mencionar que la universalidad de las RMT para la descripción de los niveles de energía de sistemas físicos 
es válida cuando la densidad media de los niveles de energía es igual a la unidad. \cite{Atas_2013}
\newline
Este trabajo consistirá en hacer un estudio espectral del hamiltoniano del modelo de Bose Hubbard 
con el objetivo de diagnosticar la región de parámetros en donde el sistema exhibe una estadística espectral 
que coincide con la propuesta por el ensabamble Gaussiano Ortogonal (GOE). El modelo a trabajar es de la forma:
\[
\hat{H}
= -\frac{J}{2} \sum_{\langle i,j\rangle} \!\left(\hat{a}_i^{\dagger}\hat{a}_j + \hat{a}_j^{\dagger}\hat{a}_i\right)
+ \frac{U}{2}\sum_{i}\hat{n}_i(\hat{n}_i - 1),
\]
El modelo de Bose Hubbard es escogido por ser un buena primera aproximación a distintos sistemas físicos de interés presentes en la física de matería condensada y física atómica, notablemente 
se han hecho realizaciones experimentales con átomos ultrafríos sobre un latice óptica y se ha podido observar la transición SF-MI (Superfluidity-Mott insulator) \cite{Zhang2010}. 
El modelo se puede emplear tanto para fermiones como para bosones. En este trabajo se modelará para bosones, y el hamiltoniano presentado describe dicha situación.
Los términos $\hat{a}_i^{\dagger},\hat{a}_j$ son los operadores creación y aniquilación respectivamente y $\hat{n}_i$ es el operador número (cuenta el número de particulas en el sitio).
El parámetro J es proporcional a la parte cinética del hamiltoniano y describe físicamente como los bosones "saltan" de un sitio a otro, el segundo parámetro es proporcional a la parte de interacción del hamiltoniano, describe 
la interacción partícula a partícula y la fuerza de interacción siendo caracterizada por U \cite{Zhang2010}. 
En este trabajo, mediante una variación de la cantidad J/U, se buscará determinar las regiones de interés.
\newline 
Para el diagnostico de las regiones de caos cuántico, se determinan las siguientes cantidades:
\begin{itemize}
    \item Mean level spacing ratio, <r>: Denotando por $E_n$ las energía ordenadas del espectro del hamiltoniano, y
    $s_n=E_{n+1}-E_{n}$, se define la cantidad:
    \[
    r_n=\frac{min(s_n,s_{n-1})}{max(s_n,s_{n-1})}
    \]
    Esta cantidad cumple con la propiedad de que la densidad media de los niveles de energía es igual a la unidad
    y por tanto es válido asumir la universalidad de RMT para en análisis.
    Se calculará para cada configuración de J/U todos los $r_n$, para posteriormente determinar el valor medio <r>.
    Se espera que este valor medio sea un indicador de caos, ya que para el ensabamble GOE, dicho valor es de <r>=0.5307 \cite{Atas_2013}.
    \item Divergencia (KL): Es una medida de diferencia entre dos distribuciones de probabilidad, en este caso, se comparará la distribucion de probabilidad de los $r_n$ 
    respecto a la distribucion de probabilidad de la misma cantidad según el ensamble GOE. Viene dada por \cite{Pausch2022}:
    \[
KL(P,P_{GOE}) = \int_{0}^{1} P(x) \, \ln \left( \frac{P(x)}{P_{GOE}(x)} \right) \, dx
\]
    En esencia, es una medida de que tanto se aleja la distribución del sistema respecto a la distribución de GOE.   
\end{itemize}
Se espera que <r> y la divergencia (KL) estén relacionadas, pues ambas son indicadores de caos cuántico. 
\newline


% Seccion pasada de Descripcion general
\textbf{OLD ver.} \newline 
\janote{Partí en párrafos esta sección, así como está no hay pausas y 
estás hablando de un montón de cosas en un sólo gran párrafo. Es dificil de 
seguir}

El estudio de sistemas cuánticos y su comportamiento\janote{a qué te referís
con ``su comportamiento''? Me suena un poco raro, no sé si fue la elección
de palabra} según las condiciones físicas\janote{igual aquí, cómo qué 
condiciones físicas?} a las que se haya sujeto 
son de importancia tanto para el ámbito aplicado como para el teórico. 
Es de interés conocer los regímenes 
en los cuales una sistema pueda exhibir una ergodicidad alta o un comportamiento fuertemente localizado para ciertos estados\janote{por qué 
es de interés? Si no vas a decir porqué, entonces cambiá la elección de 
palabras de este enunciado}.
La conjetura BGS (Bogigas-Gianoni-Schmit)\janote{Aquí, por ejemplo, hubo un 
gran salto a hablar de lleno de la conjetura BGS, esto debería ser ya otro 
párrafo} establece que el espectro de un sistema invariante antes inversiones temporales 
cuyo análogo clásico sean K sistemas\janote{cabal deberías decir qué es un 
K sistema} presenta las mismas propiedades estadísiticas que las de uno de los 
tres posibles ensambles de matrices aleatorias, GOE (Gaussian Ortogonal ensemble), GSE (Gaussian Symplectic Ensemble),
GUE ( Gaussian Unitary Ensemble). %Buscar cita
 En vista de ello, la estadística espectral es una 
herramienta apropiada para la caracterización de sistema cuánticos. \janote{No
usés $\backslash\backslash$ para hacer saltos de línea, sólo dejá una línea
de espacio para hacer saltos de párrafo, así como lo cambié aquí. Cambialo
vos en el resto del documento}

Este trabajo consistirá en hacer un estudio espectral del sistema cuántico de Bose Hubbard (con condiciones de frontera abiertas), con el motivo de identificar bajo que condiciones 
se presenta una estadística que refleje, según la conjetura BGS, las características del caos cuántico. \janote{Me parece más natural hablar aquí del Hamiltoniano del BH y luego de la r y la KL divergence}\janote{aquí no sé porqué tenés salto 
de línea, debería ser un mismo párrafo con lo siguiente}\\ 
En particular, se estudiará los cocientes entre los espaciamientos relativos de los valores propios del Hamiltoniano que describe al Bose Hubbard 
y también un correlacionador, la divergencia KL\janote{poné el nombre completo y KL como (KL)}, que permitirá inferir las regiones \janote{de parámetros} en las cuales la estadísitca del sistema 
se asemeja a la estadística del ensamble GOE\janote{toca explicar porqué 
la estadística debería parecerse a la de GOE y no a los otros ensambles}.
\janote{falta poner las ecuaciones de la r y de la KL divergence. Por eso, 
deberías agregar más contexto sobre de donde viene la r, hablar de la $P(r)$
y entonces especificar qué mide la KL divergence}
\\
El Hamiltoniano a trabajar tiene la siguiente forma:
\[
\hat{H}
= \frac{J}{2} \sum_{\langle i,j\rangle} \!\left(\hat{a}_i^{\dagger}\hat{a}_j + \hat{a}_j^{\dagger}\hat{a}_i\right)
+ \frac{U}{2}\sum_{i}\hat{n}_i(\hat{n}_i - 1),
\]
\janote{explicá los detalles de este Hamiltoniano, qué es cada símbolo, etc}
\janote{también da más contexto, que es un modelo de bosones y sitios... que sirve para modelar....}


%Seccion Actual de objetivos 
\section{Objetivos}\label{sec:objetivos}
\textbf{NEW ver.}
 \subsection{Objetivo general:}
 Estudiar la estadística espectral del Hamiltoniano de Bose Hubbard para identificar la región de paramétros sobre la cual 
 el sistema exhibe caos cuántico.
 \subsection{Objetivos específicos:}
 \begin{itemize}
    \item Estudiar el formalismo para el tratamiento de un Hamiltoniano bajo ciertas simetrías.
    \item Programar el hamiltoniano de Bose Hubbard con condiciones de frontera abiertas, sus sectores de simetría bajo el operador reflexión y las herramientas estadísticas necesarias para el estudio.
    \item Estudiar acerca de los correlacionadores propuestos y acera de la distribuciones de espaciamientos de valores propios de los ensabambles de matrices aleatorias, en particular, el ensabamble GOE.
    \item Analizar distintas configuraciones del sistema variando J/U, implementando las herramientas estádisticas y con ello identificar las regiones de interés.
 \end{itemize}
%Seccion pasada de Objetivos
\textbf{OLD ver.}
\subsection{Objetivo general}
Estudiar la estadística espectral del Hamiltoniano de Bose Hubbard para identificar bajo que condiciones\janote{no son condiciones, sino que queremos 
caracterizar la región de parámetros} 
el sistema presenta una estadísitica  que coincide con la propuesta por el ensamble de matrices aleatorias GOE (Gaussian Ortogonal Ensamble)
\subsection{Objetivos específicos}
\begin{itemize}
\item Estudiar el formalismo para el tratamiento de un Hamiltoniano bajo ciertas simetrías.
\item Sectorizar el Hamiltoniano en sus sectores de simetría par e impar.
\janote{estos primeros dos objetivos podrían condensarse en uno: estudiar el 
modelo de Bose-Hubbard y sus simetrías con condiciones de frontera abiertas.}
\item Aprender el lenguaje de programación Wolfram Mathematica, sobre el cual se programará todo el trabajo.
\item Programar la base de Fock que describe las configuraciones físicas del Bose Hubbard.
\item Programar el hamiltoniano de Bose Hubbard con condiciones de frontera abiertas.
\janote{Los últimos tres están bien resumidos por este objetivo}
\item Programar la sectorización del Hamiltoniano, bajo la simetría que existe con el operador reflexión.
\janote{este lo podés integrar al anterior, algo como ``Construir numéricamente
el Hamiltoniano de BH en el espacio de Fock completo y en los sectores de
simetría del operador de reflexión...''}
\item Estudiar acerca de las caracterizaciones espectrales que definen el caos cuántico. \janote{aquí tenés que ser más específico: estudiar sobre la 
distribución de probabilidad de los espaciamientos de niveles de matrices
aleatorias}
\item Implementar las herramientas estadísiticas.
\item Estudiar las distintas configuraciones físicas (distintos valores de J y U) del Bose Hubbard y hacer estadística para cada caso.
\item Identificar los regímenes en los cuales la estadísitica del sistema coincide con GOE.
\janote{Estos últimos tres podrías juntarlos en uno sólo parecido a: 
caracterizar el régimen caótico del modelo de BH utilizando el mean level
spacing ratio y la KL divergence}
\end{itemize}
\section{Justificación del Proyecto}
\textbf{Sección actual de Just.Pro}
\newline
El estudio de caos cuántico ha sido investigado desde hace ya más de 40 años, sin embargo debido a la innovación tecnológica y el aumento en nuestra capacidad 
computacional, se ha sido capaz de tener un estudio más robusto del tema. Este trabajo es un ejemplo de ello, ya que 
se estará trabajando con matrices cuyas dimensiones son del orden de $10^{3}$ en adelante \cite{Zhang2010}, 
pudiendo tener la capacidad de diagonalizar dichas matrices y extraer la información relevante del sistema.
Este trabajo cumple con dos objetivos, el primero es verificar que la teoría de RMT satisface en la descripción de la 
estadística espectral en las regiones de caos cuántico y el segundo objetivo es que esta primera investigación es 
un primer paso para una investigación más grande, que busca estudiar el caos cuántico y el efecto de mezclar la estadística 
de distintos sectores de simetría. Esta investigación más grande buscará arrojar más luz acerca de la descipción de sistemas cuánticos y su conexión con la teoría de RMT
\newline
\textbf{Sección pasada de Just.Pro}
Es de importancia\janote{por qué?} tener una forma cuantitativa para describir sistemas cuánticos que exhiben 
un comportamiento caótico, y ser capaz de identificar bajo que condiciones físicas se presenta 
dicho comportamiento\janote{notá que sin decir porqué es importante, tu 
enunciado no tiene sustancia}. En adición, es un tema de investigación reciente que tiene múltiples implicaciones \janote{cuáles?}
y campos posibles de investigación\janote{cuáles?}, por lo que es importante establecer una metodología óptima para el estudio posterior de otros sistemas análogos \janote{sin especificar más, este enunciado también no está diciendo mucho}.
Finalmente, en mi unidad académica, hay escaso conocimiento de estos temas de investigación, por lo que 
es crucial informar e interesar a los miembros de la comunidad\janote{pero vos no vas a hacer eso en el marco de tu proyecto de prácticas}, con el fin de poder establecer 
futuras investigaciones afines al estudio del caos cuántico \janote{este 
último enunciado lo podrías cambiar por lo que ya sabés de hacia dónde va 
tu proyecto, argumentando que este es el primer paso de un proyecto más 
grande en el que vas a investigar sobre el caos cuántico y el efecto de 
mezclar cosas de los subespacios simétricos}. 
\section{Cronograma}
(Pendiente)
\bibliographystyle{ieeetr}
\bibliography{references}
\end{document}