\documentclass[spanish,titlepage,table]{practicas}
%\usepackage[utf8]{inputenc}
\usepackage{parskip}
\usepackage[spanish]{babel}
\usepackage{hyperref}

\directlua{tab = require("src/andpocalipse")}

\usepackage[draft,inline,nomargin]{fixme} \fxsetup{theme=color}
\definecolor{jacolor}{RGB}{200,40,0}\FXRegisterAuthor{ja}{aja}{\color{jacolor}JA}
\definecolor{accolor}{HTML}{008080}\FXRegisterAuthor{ac}{anac}{\color{accolor}AC}
\definecolor{cpcolor}{RGB}{0,200,40}\FXRegisterAuthor{cp}{ancp}{\color{cpcolor}CP}

%\usepackage[mathlines]{lineno}  \linenumbers \setlength\linenumbersep{3pt}


\decimalpoint
\hypersetup{
    colorlinks,
    allcolors=blue!70!black
}
\def\ASPOR{\phantom{Asesorado por:}\hspace{.6em}}

\institute{
    Universidad de San Carlos de Guatemala\\[.5em]
    Escuela de Ciencias Físicas y Matemáticas
}
\title{
    {\LARGE Prácticas:}\\[0.5em]
    Estudio sobre los regímenes de caoticidad e integrabilidad del modelo de Bose-Hubbard 
}
\course{Licenciatura en Física Aplicada}
\author{Saúl Estuardo Nájera Allara}
\id{202107506}
\date{Septiembre de 2025}
\professor{
    M.Sc. José Alfredo de León (IF--UNAM),\\
    \ASPOR Ing. Rodolfo Samayoa (ECFM--USAC)
}

\begin{document}

\maketitle

\section{Descripción general de la institución}\label{sec:institution}
\subsection{Instituto de investigación de Ciencias Físicas y Matemáticas, USAC}
El Instituto de investigación de Ciencias Físicas y Matemáticas (ICFM) es la unidad de la Escuela de Ciencias Físicas y Matemáticas (ECFM) que promueve y realiza estudios avanzados en áreas científicas, fundamentales y aplicadas, de las ciencias físicas y matemáticas. El ICFM se proyecta como una plataforma regional de excelencia dedicada a la investigación y difusión del conocimiento en física y matemática. Las principales líneas de trabajo del ICFM son:
\begin{itemize}
    \item La investigación académica en ciencia básica y aplicada.
    \item La promoción de la investigación en ciencia básica y aplicada en el ámbito universitario.
    \item La difusión y divulgación del conocimiento generado por la investigación en ciencias físicas y matemáticas.
    \item La actualización continua de programas académicos de ciencias físicas y matemáticas.
\end{itemize}
\section{Descripción del grupo de trabjo:}
(Pendiente)
\section{Descripción general del proyecto}\label{sec:description}
El estudio de sistemas cuánticos y su comportamiento según las condiciones físicas a las que se haya sujeto 
son de importancia tanto para el ámbito aplicado como para el teórico. Es de interés conocer los regímenes 
en los cuales una sistema pueda exhibir una ergodicidad alta o un comportamiento fuertemente localizado para ciertos estados.
La conjetura BGS (Bogigas-Gianoni-Schmit) establece que el espectro de un sistema invariante antes inversiones temporales 
cuyo análogo clásico sean K sistemas presenta las mismas propiedades estadísiticas que las de uno de los 
tres posibles ensambles de matrices aleatorias, GOE (Gaussian Ortogonal ensemble), GSU ( Gaussian Symplectic Ensemble),
GUE ( Gaussian Unitary Ensemble). %Buscar cita
 En vista de ello, la estadística espectral es una 
herramienta apropiada para la caracterización de sistema cuánticos. \\
Este trabajo consistirá en hacer un estudio espectral del sistema cuántico de Bose Hubbard (con condiciones de frontera abiertas), con el motivo de identificar bajo que condiciones 
se presenta una estadística que refleje, según la conjetura BGS, las características del caos cuántico.\\
En particular, se estudiará los cocientes entre los espaciamientos relativos de los valores propios del Hamiltoniano que describe al Bose Hubbard 
y también un correlacionador, la divergencia KL, que permitirá inferir las regiones en las cuales la estadísitca del sistema 
se asemeja a la estadística del ensamble GOE.
\\
El Hamiltoniano a trabajar tiene la siguiente forma:
\[
\hat{H}
= \frac{J}{2} \sum_{\langle i,j\rangle} \!\left(\hat{a}_i^{\dagger}\hat{a}_j + \hat{a}_j^{\dagger}\hat{a}_i\right)
+ \frac{U}{2}\sum_{i}\hat{n}_i(\hat{n}_i - 1),
\]
\section{Objetivos}\label{sec:objetivos}
\subsection{Objetivo general}
Estudiar la estadísitica espectral del Hamiltoniano de Bose Hubbard para identificar bajo que condiciones 
el sistema presenta una estadísitica  que coincide con la propuesta por el ensamble de matrices aleatorias GOE (Gaussian Ortogonal Ensamble)
\subsection{Objetivos específicos}
\begin{itemize}
\item Estudiar el formalismo para el tratamiento de un Hamiltoniano bajo ciertas simetrías.
\item Sectorizar el Hamiltoniano en sus sectores de simetría par e impar.
\item Aprender el lenguaje de programación Wolfram Mathematica, sobre el cual se programará todo el trabajo.
\item Programar la base de Fock que describe las configuraciones físicas del Bose Hubbard.
\item Programar el hamiltoniano de Bose Hubbard con condiciones de frontera abiertas.
\item Programar la sectorización del Hamiltoniano, bajo la simetría que existe con el operador reflexión.
\item Estudiar acerca de las caracterizaciones espectrales que definen el caos cuántico.
\item Implementar las herramientas estadísiticas.
\item Estudiar las distintas configuraciones físicas (distintos valores de J y U) del Bose Hubbard y hacer estadística para cada caso.
\item Identificar los regímenes en los cuales la estadísitica del sistema coincide con GOE.
\end{itemize}
\section{Justificación del Proyecto}
Es de importancia tener una forma cuantitativa para describir sistemas cuánticos que exhiben 
un comportamiento caótico, y ser capaz de identificar bajo que condiciones físicas se presenta 
dicho comportamiento. En adición, es un tema de investigación reciente que tiene múltiples implicaciones 
y campos posibles de investigación, por lo que es importante establecer una metodología óptima para el estudio posterior de otros sistemas análogos.
Finalmente, en mi unidad académica, hay escaso conocimiento de estos temas de investigación, por lo que 
es crucial informar e interesar a los miembros de la comunidad, con el fin de poder establecer 
futuras investigaciones afines al estudio del caos cuántico.
\section{Cronograma}
(Pendiente)
\end{document}