\documentclass[spanish,titlepage,table]{practicas}
%\usepackage[utf8]{inputenc}
\usepackage{parskip}
\usepackage[spanish]{babel}
\usepackage{hyperref}
\usepackage{physics}

\directlua{tab = require("src/andpocalipse")}

\usepackage[draft,inline,nomargin]{fixme} \fxsetup{theme=color}
% Comentarios de JA 
\definecolor{jacolor}{RGB}{200,40,0}
\FXRegisterAuthor{ja}{aja}{\color{jacolor}JA}
% Comentarios de Saul
\definecolor{scolor}{HTML}{008080}
\FXRegisterAuthor{s}{asn}{\color{scolor}S}

%\usepackage[mathlines]{lineno}  \linenumbers \setlength\linenumbersep{3pt}


\decimalpoint
\hypersetup{
    colorlinks,
    allcolors=blue!70!black
}
\def\ASPOR{\phantom{Asesorado por:}\hspace{.6em}}

\institute{
    Universidad de San Carlos de Guatemala\\[.5em]
    Escuela de Ciencias Físicas y Matemáticas
}
\title{
    {\LARGE Prácticas:}\\[0.5em]
   Caos cuántico en el modelo de\\[0.4em]
	 Bose-Hubbard 
}
\course{Licenciatura en Física Aplicada}
\author{Saúl Estuardo Nájera Allara}
\id{202107506}
\date{Septiembre de 2025}
\professor{
    M.Sc. José Alfredo de León (IF--UNAM),\\
    \ASPOR Ing. Rodolfo Samayoa (ECFM--USAC)
}

\begin{document}

\maketitle

\section{Descripción general de la institución}\label{sec:institution}
\subsection{Instituto de investigación de Ciencias Físicas y Matemáticas, USAC} 
El Instituto de investigación de Ciencias Físicas y Matemáticas (ICFM) es la unidad de la Escuela de Ciencias Físicas y Matemáticas (ECFM) que promueve y realiza estudios avanzados en áreas científicas, fundamentales y aplicadas, de las ciencias físicas y matemáticas. El ICFM se proyecta como una plataforma regional de excelencia dedicada a la investigación y difusión del conocimiento en física y matemática. Las principales líneas de trabajo del ICFM son:
\begin{itemize}
    \item La investigación académica en ciencia básica y aplicada.
    \item La promoción de la investigación en ciencia básica y aplicada en el ámbito universitario.
    \item La difusión y divulgación del conocimiento generado por la investigación en ciencias físicas y matemáticas.
    \item La actualización continua de programas académicos de ciencias físicas y matemáticas.
\end{itemize}

%Seccion  actual de descripcion de trabajo
\section{Descripción general del proyecto}\label{sec:description}
El estudio estadístico del espectro de sistemas cuánticos es una herramienta de valiosa utilidad para el ámbito teórico, experimental y aplicado.
Se han distinguido dos tipos de propiedades: globales y locales.
Respecto a las propiedades locales, se han estudiado fluctuaciones del espectro de energía, es decir, se han observado desviaciones de las distribuciones de niveles de energía respecto de un comportamiento promedio~\cite{Bohigas_1984}.


En un contexto de física clásica, el caos presenta dos características. 
La primera es sensibilidad exponencial a cambios pequeños en 
las condiciones iniciales y la segunda es el ``\textit{mixing}'', cualquier distribución inicial de condiciones iniciales se dispersa hasta volverse indistinguible de la distribución de equilibrio. 
En física cuántica, en vista del principio de incertidumbre, no es posible tener 
trayectorias definidas en el espacio de fase. Una de las formas para estudiar el caos cuántico es mediante la estadística espectral, es decir, estudiando el espectro de energía del Hamiltoniano asociado al sistema.

Las fluctuaciones espectrales de sistemas cuánticos complejos es analizada mediante 
el marco teórico de la Teoría de Matrices Aleatorias (RMT) en varias áreas de la física, 
tales como la física de materia condensada, física nuclear y física atómica.
Las fluctuaciones contienen señales distintivas para la identificación de diferentes fases observadas 
en sistemas físicos, por ejemplo para identificar límites entre integrabilidad o caoticidad en un sistema con análogo clásico,
determinación de fases de metalicidad o insulación, fases localizadas o termalizadas de sistemas de muchos cuerpos, estudio de regímenes en espectros nucleares, entre otros~\cite{Tekur2020}
El presente consenso es que las fluctuaciones espectrales de sistemas cuánticos complejos presentan 
una repulsión entre sus niveles de energía que están en concordancia con alguno de los ensambles de matrices 
aleatorias, en los que se resaltan los ensambles GOE (Gaussian Orthogonal Ensemble), GUE (Gaussian Unitary Ensemble) y GSU (Gaussian Simplectic Ensemble).
El primero en usar la idea de RMTs fue Wigner, en los años 50, con el motivo de estudiar el espectro de núcleos atómicos pesados~\cite{Wigner1955}.
Una de las observaciones que resaltan de su discusión es la observación de que los niveles de energía presentan una tendencia a repelerse unos a otros, 
ello motivo en parte a el uso de RMT para la descripción del hamiltoniano de este tipo de sistemas. 

Para hamiltonianos cuánticos cuya contraparte clásica es integrable, la conjetura Berry-Tabor establece que 
la estadística de los niveles de energía sigue una estadística de Poisson~\cite{BerryTabor1977}, mientras que para hamiltonianos 
cuya contraparte clásica sea caótica, la conjetura BGS (Bogigas, Gianoni, Schmit) establece que la estadística de los niveles de energía es descrita 
por alguno de los ensambles de RMT~\cite{Atas_2013}
Es de mencionar que la universalidad de RMT para la descripción de los niveles de energía de sistemas físicos 
es válida cuando la densidad media de los niveles de energía es igual a la unidad~\cite{Atas_2013}.


Este trabajo consistirá en hacer un estudio espectral del hamiltoniano del modelo de Bose Hubbard 
con el objetivo de diagnosticar la región de parámetros en donde el sistema exhibe una estadística espectral 
que coincide con la propuesta por el ensamble Gaussiano Ortogonal (GOE). El modelo de Bose Hubbard es escogido por ser un buena primera aproximación a distintos sistemas físicos de interés presentes en la física de matería condensada y física atómica, notablemente, 
se han hecho realizaciones experimentales con átomos ultrafríos sobre un red óptica y se ha podido observar la transición SF-MI (Superfluidez-Aislante de Mott)~\cite{Zhang2010}. 
El modelo se puede emplear tanto para fermiones como para bosones. En este trabajo se modelará para bosones. El Hamiltoniano de este modelo es:
\begin{equation}
\hat{H}
= -\frac{J}{2} \sum_{\langle i,j\rangle} \!\left(\hat{a}_i^{\dagger}\hat{a}_j + \hat{a}_j^{\dagger}\hat{a}_i\right)
+ \frac{U}{2}\sum_{i}\hat{n}_i(\hat{n}_i - 1).
\end{equation}
Los términos $\hat{a}_i^{\dagger},\hat{a}_j$ son los operadores creación y aniquilación respectivamente y $\hat{n}_i$ es el operador número (cuenta el número de particulas en el sitio).
El parámetro $J$ describe la parte cinética del Hamiltoniano y dicta la probabilidad de que los bosones ``salten'' de un sitio a otro adyacente. El segundo parámetro $U$ describe la parte de interacción del Hamiltoniano, nos dice cómo  
es la fuerza de interacción entre bosones en mismo sitio~\cite{Zhang2010}. 
En este trabajo, mediante una variación de la cantidad $J/U$, se buscará determinar las regiones de integrabilidad y caos del sistema.


Para el diagnostico de las regiones de caos cuántico, se utilizarán las siguientes cantidades:
\begin{itemize}
    \item Mean level spacing ratio, $\ev{r}$:

Denotando por $E_n$ las energía ordenadas del espectro del hamiltoniano, y el espaciamiento consecutivo de niveles de energía como 
    $s_n=E_{n+1}-E_{n}$, se define la cantidad:
		\begin{equation}
			r_n=\frac{\min(s_n,s_{n-1})}{\max(s_n,s_{n-1})}.
		\end{equation}
    Esta cantidad cumple con la propiedad de que la densidad media de los niveles de energía es igual a la unidad
    y por tanto es válido asumir la universalidad de RMT para el análisis.
    Estudiando la distribución de probabilidad asociada a las r's, es posible determinar si la configuración está en un regímen integrable 
    o caótico. Por lo que es importante tener presente la forma análitica de la distribución de probabilidad de los cocientes entre 
    los espaciamientos de niveles del ensabamble GOE y de Poisson, mediante $\ev{r}$ es posible caracterizar cada 
    regímen. En particular, para el ensamble GOE, dicho valor es de $\ev{r}_{GOE}=0.5307$~\cite{Atas_2013} y para un regímen integrable un valor de $\ev{r}_{Poisson}=0.38629$~\cite{Atas_2013}. 
    La distribución asociada a GOE es de la forma,($\Theta(r)$ es la función de Heaviside)~\cite{Atas_2013}:
     \snote{Listo}\janote{ojo que ambas distribuciones de 
las dos siguientes ecuaciones les hace falta una Heaviside multiplicando y 
tal vez un factor, No recuerdo bien, porfa revisá bien Atas et al}:
\begin{equation}
    P_{GOE}(r)=2\Theta(1-r)\cdot(\frac{27}{8}\frac{r+r^{2}}{(1+r+r^{2})^{5/2}}).
\end{equation}
Y para la distribución asociada a un espectro integrable (Poisson) es de la forma~\cite{Tekur2020}:
\begin{equation}
    P_{Poisson}(r)=2\Theta(1-r)\cdot(\frac{1}{(1+r)^{2}})
\end{equation}
    Se calculará para cada configuración de $J/U$ todos los $r_n$, para posteriormente determinar el valor medio $\ev{r}$.

    \item Divergencia de Kullback-Leibler ($KL$): Es una medida de diferencia entre dos distribuciones de probabilidad. En este caso, se comparará la distribucion de probabilidad de los $r_n$ de la ecuación (2)
    respecto a la distribucion de probabilidad de la misma cantidad según el ensamble GOE. Viene dada por~\cite{Pausch2022}:
    \begin{equation}
KL(P,P_{GOE}) = \int_{0}^{1} P(x) \, \ln \left( \frac{P(x)}{P_{GOE}(x)} \right) \, dx.
    \end{equation}
    En esencia, es una medida de que tanto se aleja la distribución del sistema respecto a la distribución de GOE.   
\end{itemize}
Tanto $\ev{r}$ como la divergencia $KL$ son indicadores que permiten establecer una conexión entre las fluctuaciones espectrales de un sistema físico 
con la universalidad predicha por RMT, y por tanto permiten establecer una conexión con la definción de caos cuántico adoptada en este trabajo.

\janote{Esto integralo en el bullet point de la $\ev{r}$}
\snote{Sección trasaldada a donde está $\ev{r}$ }
\janote{Quiero que hagás una integración maś suave, no sólo mover el texto. 
De hecho, es más natural definir las $P(r)$ antes de decir los valores de 
$\ev{r}$, porque podés mencionar que usando esas distribuciones uno puede 
mostrar que el valor esperado son los numeritos tales.. }\snote{Trasladado suave hecho.}

%Seccion Actual de objetivos 
\section{Objetivos}\label{sec:objetivos}
\subsection{Objetivo general}
  Estudiar la estadística espectral del Hamiltoniano de Bose-Hubbard para identificar la región de paramétros sobre la cual 
 el sistema exhibe caos cuántico.
 \subsection{Objetivos específicos}
  \begin{itemize}
    \item Estudiar el modelo de Bose-Hubbard e investigar su importancia en la física contemporánea.
    \item Estudiar acerca del mean level spacing ratio $\ev{r}$ y la divergencia $KL$. También investigar acerca de la distribuciones de los cocientes de  \snote{Listo.}\janote{los cocientes de} espaciamientos de valores propios de los ensambles de matrices aleatorias, en particular, el ensamble GOE.
    \item Programar el Hamiltoniano de Bose-Hubbard con condiciones de frontera abiertas, sus sectores de simetría bajo el operador reflexión y los dos indicadores de caos propuestos.
    \item Analizar distintas configuraciones del sistema variando $J/U$ con los indicadores de caos propuestos y con ello identificar las regiones de integrabilidad y caos del sistema.
 \end{itemize}
 
\section{Justificación del Proyecto}
El estudio de caos cuántico ha sido investigado desde hace ya más de 40 años, notablemente empezando con un estudio de sistemas cuánticos
que tuviesen una contraparte clásica~\cite{Bohigas_1984}. Gracias a la innovación tecnológica y al aumento en la capacidad 
computacional, se ha sido capaz de tener un estudio más robusto del tema,
pudiendo estudiar con mayor facilidad sistemas de muchos cuerpos que no tienen contraparte clásica, algo que ha traido un reavivamiento en el estudio de caos cuántico.\snote{Agregué y quité cosas, me parece que está mejor estructurado}\janote{este enunciado partilo en dos, haciendo las modificaciones necesrias}. Este trabajo es un ejemplo de ello, ya que se estará trabajando con 
un sistema que no tiene análogo clásico y es un buen punto de partida para estudiar el caos cuántico desde el punto de vista de RMT.
Este trabajo es un primer paso para una investigación más grande, que busca estudiar el caos cuántico y el efecto de mezclar la estadística 
de distintos sectores de simetría. Esta investigación más grande buscará arrojar más luz acerca de la descipción de sistemas cuánticos y su conexión con la teoría de RMT.

\section{Cronograma}
A continuación, la lista de tarea a llevar a cabo a lo largo de la práctica:
\begin{itemize}
    \item Tarea 1: Aprender y profundizar acerca de las simetrías en un hamiltoniano y su sectorización.
    \item Tarea 2: Aprender lo básico acerca del lenguaje de programación Wolfram Mathematica.
    \item Tarea 3: Programar la base de fock que describe las configuraciones físicas del modelo.
    \item Tarea 4: Programar el hamiltoniano de Bose Hubbard con condiciones de frontera abiertas.
    \item Tarea 5: Programar los sectores de simetría del hamiltoniano.
    \item Tarea 6: Programar las herramienta de diagnostico de caos cuántico. 
    \item Tarea 7: Obtener los valores propios del hamiltoniano para distintos valores de $J/U$ y crear una base de datos con los mismos.
    \item Tarea 8: Hacer la estadística respectiva para cada configuración calculada.
    \item Tarea 9: Presentar los datos obtenidos de tal forma que sea clara la exposción de información de diagnostico.
\end{itemize}
\newcommand{\Pcolor}{\cellcolor[HTML]{C0C0C0}}%
\def\ANDpocalipse{&&&&&&&&&&&&&&&&&&&&&&&&}
%
\begin{center}
{
\renewcommand{\arraystretch}{1.75}
\begin{tabular}{|c|*{16}{c|}} \hline
\textbf{Tareas} & \multicolumn{4}{c|}{\textbf{Mes 1}} & \multicolumn{4}{c|}{\textbf{Mes 2}} & \multicolumn{4}{c|}{\textbf{Mes 3}} & \multicolumn{4}{c|}{\textbf{Mes 4}} \\ \hline
Tarea 1&\directlua{tab.ANDpocalipse({1,2})}\\
\hline
Tarea 2&\directlua{tab.ANDpocalipse({1,2,3,4})}\\
\hline
Tarea 3&\directlua{tab.ANDpocalipse({4,5})}\\
\hline
Tarea 4&\directlua{tab.ANDpocalipse({6,7})}\\
\hline
Tarea 5&\directlua{tab.ANDpocalipse({8,9})}\\
\hline
Tarea 6&\directlua{tab.ANDpocalipse({10})}\\
\hline
Tarea 7&\directlua{tab.ANDpocalipse({11,12})}\\
\hline
Tarea 8&\directlua{tab.ANDpocalipse({13,14})}\\
\hline
Tarea 9&\directlua{tab.ANDpocalipse({15,16})}\\
\hline
\end{tabular}%
}
\end{center}
\bibliographystyle{ieeetr}
\bibliography{references}
\end{document}